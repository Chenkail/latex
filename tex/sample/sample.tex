\documentclass[12pt]{article}

\input{../preamble}


\begin{document}
\title{Sample TeX File}
\author{Your name here}
\date{}

\maketitle
% \pagebreak

\tableofcontents{}
\pagebreak

% -------- Begin main content --------
\section{This is a section}
Here is a list:
\begin{itemize}[noitemsep, topsep=0pt, label=-]
\item Item 1
\item Item 2
\end{itemize}

\gap

\subsection{Here is a subsection}
\begin{proof} \quad % Blank line
\begin{spreadlines}{2\baselineskip}
\begin{enumerate}[label=(\alph*)]
\item Numbered item 1
\item Numbered item 2
\end{enumerate}
\end{spreadlines}
\end{proof}

Use \textbackslash{}mathrm to make normal text in a math block, such as the d in the integral below:

\begin{align*}
\int_{0}^{\frac{\pi}{2}} \; \sin(x) \, \mathrm{d}x &= -\cos(x) \Big|_0^\frac{\pi}{2}
\end{align*}

\gap

\subsubsection{Here is a subsubsection}
You can align equations as follows:
\begin{align}
x&=y\\
x^2&=y^2
\end{align}

You can also remove the equation numbers by adding an asterisk:
% \addtolength{\jot}{\jot}
\begin{align*}
x&=\frac{1}{2}\\
x^2&=\parens{\frac{1}{2}}^2
\end{align*}

\begin{lemma}
This is a lemma
\end{lemma}

$\sumto{n}{1}{\infty}$

\section{Sample table}

\begin{center}
\begin{tabulary}{\linewidth}{LLCL}
\hline
{\bfseries Name} &
{\bfseries Meaning} &
{\bfseries Symbol} &
{\bfseries LaTeX}\\\hline
Empty set & The set containing zero elements & ${\emptyset}$ OR \{\} & {\textbackslash}emptyset\\\hline

In & a is an element of b & a ${\in}$ b & {\textbackslash}in\\\hline

Not in & a is not an element of b & a ${\notin}$ b & ~ \\\hline

Subset & All elements of a are in b & a ${\subseteq}$ b & {\textbackslash}subseteq\\\hline

Proper subset & A is a subset of b but not equal to b & a ${\subset}$ b & {\textbackslash}subset\\\hline

Universal set & Set of all possible elements & U & ~ \\\hline

Union & Elements in either A or B or both & A ${\cup}$ B & ~ \\\hline

Intersection & Elements in both A and B & A ${\cap}$ B & ~ \\\hline

Set difference & Elements in A that are not in B & A - B & ~ \\\hline

Complement (sets) & Set difference U - A
 & \=A or A\textsuperscript{c} & ~ \\\hline

Power set & Set of all possible subsets of A & P(A) & ~ \\\hline

Cardinality & Number of distinct elements in A & {\textbar}A{\textbar} or card(A) & ~ \\
\hline
\end{tabulary}
\end{center}

\end{document}
